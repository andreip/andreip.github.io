%______________________________________________________________________________________________________________________
% @brief    LaTeX2e Resume for Andrei-Niculae Petre
\documentclass[margin,line]{resume}

\RequirePackage{color}
%Setup hyperref package, and colours for links
\usepackage{hyperref}
\definecolor{linkcolour}{rgb}{0,0.2,0.6}
\hypersetup{colorlinks,breaklinks,urlcolor=linkcolour, linkcolor=linkcolour}

%______________________________________________________________________________________________________________________
\begin{document}
\name{\Large Andrei-Niculae PETRE}
\begin{resume}

    %__________________________________________________________________________________________________________________
    % Contact Information
    \section{\mysidestyle Contact\\Information}

	phone: +40 727 029 961              \\
	e-mail:  \href{mailto:p31andrei@gmail.com}{p31andrei@gmail.com}  \vspace{0mm}\\\vspace{-4.5mm}\\%
	website: \href{http://andreipetre.ro}{andreipetre.ro}

    %__________________________________________________________________________________________________________________
    % Education
    \section{\mysidestyle Education}

	\textbf{University POLITEHNICA of Bucharest}, Romania \hfill \textbf{ Oct 2010 -- present}\vspace{1.2mm}\\%
	\textsl{\textbf{BSc}, Major in Engineering and Computer Science (Class of 2014)}\vspace{1.5mm}\\
	\small{\textbf{Faculty of Automatic Control and Computer Science}}

    %__________________________________________________________________________________________________________________
    % Professional Experience
    \section{\mysidestyle Professional\\Experience}

	\vspace{1.2mm}{\fontsize{4mm}{0em}\selectfont
					\textbf{University POLITEHNICA of Bucharest}, Romania
				  }
		\hfill \textbf{Oct 2013 -- May 2014}\vspace{1mm}\\
	{\fontsize{4mm}{0em}\selectfont
		\textsl{Teaching Assistant}\vspace{1.5mm}\\
	}
	Classes: Data Structures, Algorithms Analysis, Programming Paradigms.\vspace{1.2mm}\\
	Reference to:
	\begin{list2}
		\item \textbf{Matei Popovici}, Analysis of Algorithms Professor, \href{mailto:pdmatei@gmail.com}{pdmatei@gmail.com}
		\item \textbf{Alex Olteanu}, Data Structures Professor, \href{mailto:alex.c.olteanu@gmail.com}{alex.c.olteanu@gmail.com}
	\end{list2}

	\vspace{1.2mm}{\fontsize{4mm}{0em}\selectfont
					\textbf{uberVU}, Social Media Analytics
				  }
		\hfill \textbf{Jun 2012 -- Sept 2013}\vspace{1.2mm}\\
	{\fontsize{4mm}{0em}\selectfont
		\textsl{Software Developer}\vspace{1.5mm}\\
	}
	I had both frontend and backend responsabilities, working with a small team in a startup agile environment, on the edge of releases. We all wrote tests, offered code reviews, designed projects, following extreme programming habits.\vspace{1.2mm}\\
	Frontend: internal framework, django server, mysql, database migrations, deployment script\vspace{1.2mm}\\
	Backend: customize data acquisition controllers from pipeline, elasticsearch API\vspace{1.2mm}\\
	Languages used: python, coffeescript\vspace{1.2mm}\\
	Reference to:
	\begin{list2}
		\item \textbf{Bogdan Sandulescu}, uberVU CTO, \href{mailto:bogdans83@gmail.com}{bogdans83@gmail.com}
		\item \textbf{Andrei Ismail} uberVU Software Developer, \href{mailto:iandrei@gmail.com}{iandrei@gmail.com}
	\end{list2}

	\vspace{1.2mm}{\fontsize{4mm}{0em}\selectfont
					\textbf{Artificial Intelligence \& Multi-Agent Systems Lab}
				  }
		\hfill \textbf{Jun 2011 -- Sept 2011}\vspace{1.2mm}\\
	\vspace{1mm}{\fontsize{4mm}{0em}\selectfont
					\textsl{Intern, AI Class homework design}\vspace{1.5mm}\\
				}
	I developed a homework for the AI class with a robot exploring a space. As a 1st year student, the declarative way of thinking from prolog was a challenging and wonderful experience. Customized a Linux minimal VM, as a sandbox, to run over an open-source automated homework evaluation utility, \href{https://elf.cs.pub.ro/vmchecker/ui/?locale=en}{\textbf{Vmchecker}}.\vspace{1.2mm}\\
	Languages used: prolog, python, bash\vspace{1.2mm}\\
	Reference to \textbf{Andrei Ismail}, PhD Student, University POLITEHNICA of Bucharest, \href{mailto:iandrei@gmail.com}{iandrei@gmail.com}

    %__________________________________________________________________________________________________________________
    % Computer Skills
    \section{\mysidestyle Computer\\Skills}

	Python, Coffeescript: industry experience\\
	C, Java: school and personal projects\\
	Algorithms and Data Structures: school projects, programming contests\\
	Linux: LPIC-1, operating systems classes (user,kernel space)

    %__________________________________________________________________________________________________________________
    % Other Activities

    \section{\mysidestyle Other\\Activities}

	\textbf{ROSEdu} (member Aug 2011 - present) is a community of
	programming and open source software enthusiasts in the educational environment. Our mission is to
	initiate, support and develop education based on the values of open source.\vspace{1mm}
	Throughout the years I have been involved in many projects:
	\begin{list2}
        \item \textbf{coordinated} the \textbf{Open Source Development Course}, for two consecutive editions
        \item offered \textbf{mentorship} and \textbf{held technical presentations} (git, Flask) for the \textit{Open Source Development Course}
        \item offered \textbf{mentorship} in a course dedicated to teaching high school teachers Python programming language, so they can afterwards teach it in class
        \item part of a team involved in writing a Python book addressed to high school and beginners in programming
	\end{list2}\vspace{-3mm}
	\small{Reference to \textbf{Razvan Deaconescu}, ROSEdu Founder, \href{mailto:razvan.deaconescu@cs.pub.ro}{razvan.deaconescu@cs.pub.ro}}

	\textbf{National Romanian Scouts} (member 2007 - present)\vspace{1mm}\\
	Being a scout means to me to always be happy, friendly, play games, hike, camp, as well as make new friends and be a good example for those alongside.

	Hobbies: I very much like to swim. From time to time, I hike on Romanian mountains. Also, in my spare time, I like to improve my chess game.



%______________________________________________________________________________________________________________________
\end{resume}
\end{document}


%______________________________________________________________________________________________________________________
% EOF

